\documentclass[11pt]{article}
\usepackage{cs170}

\def\title{Homework 10}
\def\duedate{4/12/2023, at 10:00 pm (grace period until 11:59pm)}

\begin{document}
\maketitle

Due \textbf{\duedate}

\question{Study Group}
List the names and SIDs of the members in your study group.
If you have no collaborators, you must explicitly write ``none''.

\begin{solution} I worked on this homework with the following collaborators:
\begin{itemize}
    \item Lakshya Nagal, SID: 3037935253
\end{itemize}
\end{solution}

\question{Flow vs LP}
You play a middleman in a market of $m$ suppliers
and $n$ purchasers.
The $i$-th supplier can supply up to $s[i]$ products,
and the $j$-th purchaser would like to buy up to $b[j]$ products.\\

\noindent However, due to legislation, supplier $i$ can only sell  to a purchaser $j$ if they
are situated at most 1000 miles apart.
Assume that you're given a list $L$ of all the pairs $(i,j)$ such that
supplier $i$ is within $1000$ miles of purchaser $j$.
Given $m, \, n,\, s[1..m], \, b[1..n]$, and $L$ as input,
your job is to compute the maximum number of products that can be 
sold.
The runtime of your algorithm must be polynomial in $m$ and $n$.\\

\noindent For parts (a) and (b), assume the product is divisible—that is, it's OK to sell a fraction of a product. 

\begin{subparts}
\subpart
Show how to solve this problem, using
a network flow algorithm as a subroutine.
Describe the graph
and explain why the output from the network flow algorithm
gives a valid solution to this problem.\\
\begin{solution}\\
    We begin by setting a up a graph with $2+m+n$ nodes:
    \begin{itemize}
        \item One source node.
        \item One sink node.
        \item $m$ nodes, one for each supplier.
        \item $n$ nodes, one for each purchaser.
    \end{itemize}
    For the edges:
    \begin{itemize}
        \item $\forall(i, j)\in L$ draw an edge from the $i^{th} ~m$ node to the $j^{th} ~n$ node with capacity
        $s[i]$. 
        \item Draw an edge with capacity $s[i]$ from the source to the $i^{th}$ supplier.
        \item Draw an edge wih capacity $b[j]$ from the $j^{th}$ purchaser to the sink node.
    \end{itemize}
    This gives a valid solution because for every supplier it well send flow to the corresponding purchaser equal to the 
    max amount of products that the purchaser can buy. Additionally, the max-flow algorithm will saturate the all possible edges, and guarantee 
    the max-flow is outputted, which corresponds to the max amount of products that can be sold. 
\end{solution}
\subpart Formulate this as a linear program.
Explain why this correctly solves the problem, and the LP can be solved in polynomial time.\\
\begin{solution}
    Define $p(i, j)$ be the number of products bought by purchaser $i$ from the $j^{th}$ supplier.
    \begin{align*}
        \textbf{maximize: }\sum_{i=1}^{m} &\sum_{j=1}^{n}p(i, j)\\
        \textbf{Subject to: } &\sum_{i=1}^{m}p(i, j) \le b[j], \text{ where } j \in [1,\dots, n]\\
        &\sum_{j=1}^{n} p(i, j) \le s[i],  \text{ where } i \in [1, \dots, m]\\
        &\forall (i, j) \in L, p(i, j) \geq 0\\
        &\forall (i, j) \not \in L, p(i, j) = 0
    \end{align*}
    This essentially model the graph described in part (a), where we want to maximize the amount of products
    sold to purchaser, under the constraints that we don't exceed the amount of items the purchasers can buy.
    There are $mn$ variables and at most $mn$ constraints, and the runtime is polynomial in terms of $n$ and $m$.
\end{solution}

\subpart Now let's assume you \textit{cannot} sell a fraction of a product.
In other words, the number of products sold by each supplier to
each purchaser must be an integer.
Which formulation would be better, network flow or linear programming?
Explain your answer.
\begin{solution}
    The linear program above is vulnerable to finding non integer solutions, so the best approach would be to run a network flow algorithm like Ford-Fulkerson.
    Since our capacities are all integers, a network flow algorithm will find an integer solution.
\end{solution}
\end{subparts}

\newpage
\question{Reduction to 3-Coloring}

Given a graph $G = (V, E)$, a valid 3-coloring assigns each vertex in the graph a color from \{red, green, blue\} such that for any edge $(u, v)$, $u$ and $v$ have different colors. In the 3-coloring problem, our goal is to find a valid 3-coloring if one exists. In this problem, we will give a reduction from 3-SAT to the 3-coloring problem. Since we know that 3-SAT is NP-Hard (there is a reduction to 3-SAT from every NP problem), this will show that 3-coloring is NP-Hard (there is a reduction to 3-coloring from every NP problem).

\noindent In our reduction, the graph will start with three special vertices, labelled $v_{\textsf{TRUE}}$, $v_{\textsf{FALSE}}$, and $v_{\textsf{BASE}}$, as well as the edges $(v_{\textsf{TRUE}}, v_{\textsf{FALSE}})$, $(v_{\textsf{TRUE}}, v_{\textsf{BASE}})$, and $(v_{\textsf{FALSE}}, v_{\textsf{BASE}})$. 
\begin{subparts}

\subpart For each variable $x_i$ in a 3-SAT formula, we will create a
pair of vertices labeled $x_i$ and $\lnot x_i$. How should we add
edges to the graph such that in any valid 3-coloring, one of
$x_i, \lnot x_i$ is assigned the same color as $v_{\textsf{TRUE}}$ and the other is assigned the same color as $v_{\textsf{FALSE}}$? 

\emph{Hint: any vertex adjacent to $v_{\textsf{BASE}}$ must have the same color as either $v_{\textsf{TRUE}}$ or $v_{\textsf{FALSE}}$. Why is this?}\\
\begin{solution}
    We create an edge from $x_i$ to $\lnot x_i$, in this way the two vertices are adjacent (must be colored differently), and we assign one with $v_{TRUE}$ and one with $v_{FALSE}$. We can then proceed
    to connect all other vertices to these vertices and assign them to $v_{BASE}$.
\end{solution}

\subpart Consider the following graph, which we will call a ``gadget'':

\begin{center}
\begin{tikzpicture}[
node distance =6mm and 12mm,
C/.style = {circle, draw, minimum size=1em}
                    ] 
\node[fill=gray!40] (u1) [C] {$v_9$};
\node (u2) [C, above right=of u1] {$v_7$};
\node (u3) [C, below right=of u1] {$v_8$};
\node (u4) [C, right=of u2] {$v_6$};
\node (u5) [C, above right=of u4] {$v_4$};
\node (u6) [C, below right=of u4] {$v_5$};
\node[fill=gray!40] (u7) [C, right=of u5] {$v_1$};
\node[fill=gray!40] (u8) [C, right=of u6] {$v_2$};
\node[fill=gray!40] (u9) [C, right=4.95cm of u3] {$v_3$};

\draw (u1) -- (u2);
\draw (u1) -- (u3);
\draw (u2) -- (u3);
\draw (u2) -- (u4);
\draw (u3) -- (u9);
\draw (u4) -- (u5);
\draw (u4) -- (u6);
\draw (u5) -- (u7);
\draw (u5) -- (u6);
\draw (u6) -- (u8);
    \end{tikzpicture}
    \end{center}

Consider any valid 3-coloring of this graph that does \textit{not} assign the color red to any of the gray vertices ($v_1, v_2, v_3, v_9)$. Show that if $v_9$ is assigned the color blue, then at least one of $\{v_1, v_2, v_3\}$ is assigned the color blue.

\emph{Hint: it's easier to prove the contrapositive!}\\
\begin{solution}\\
    \textbf{Contrapositive:} If $\lnot \{v_1, v_2, v_3\}$ is assigned blue $\Rightarrow ~v_9$ is not assigned blue.\\
    \textbf{Proof by Contrapositive: }If none of $v_1, v_2, v_3$ are assigned color blue and red is not assigned to any of the gray vertices, then $v_1, v_2, v_3$ must be green. Consequently $v_4, v_5$ must be assigned to combination of either blue or red, 
    forcing $v_6$ to be green. $v_8$ and $v_7$ must also be assigned a combination of blue or red, forcing $v_9$ to be green. Therefore if none of $\{v_1, v_2, v_3\}$ are assigned blue, then $v_9$ is not assigned blue. Which proves the 
    contrapositive, so our original claim must be true.
\end{solution}

\subpart We have now observed the following about the graph we are creating in the reduction:
\begin{enumerate}[(i)]
\item For any vertex, if we have the edges $(u, v_{\textsf{FALSE}})$ and $(u, v_{\textsf{BASE}})$ in the graph, then in any valid 3-coloring $u$ will be assigned the same color as $v_{\textsf{TRUE}}$. 

\item Through brute force one can also show that in a gadget, if all the following hold:
\begin{enumerate}[(1)]
\item All gray vertices are assigned the color  green or blue.
\item $v_9$ is assigned the color blue.
\item At least one of $\{v_1, v_2, v_3\}$ is assigned the color blue.
\end{enumerate}
Then there is a valid coloring for the white vertices in the gadget. 
\end{enumerate}

Using these observations and your answers to the previous parts, \textbf{give
a reduction from 3-SAT to 3-coloring. Prove that your reduction is correct (you do not need to prove any of the observations above). }

\emph{Hint: create a new gadget per clause!}\\
\begin{solution}\\
    \textbf{Reduction Algorithm } 3-SAT to 3-COLORING:
    \begin{itemize}
        \item[1.] $\forall x_i$ in a 3-SAT formula, draw an edge from $x_i$ to $\lnot x_i$ if it exist. Assign $x_i$ and $\lnot x_i$ to one of $v_{TRUE}$ or $v_{FALSE}$ such that $x_i$ and $\lnot x_i$ are different.
        \item[2.] Connect $x_i$ and $\lnot x_i$ to other variables assigned to $v_{BASE}$
    \end{itemize}
    \textbf{Proof $\rightarrow$: } \textbf{If $\exists$ a 3-SAT solution $\Rightarrow ~\exists$ a 3-COLORING of the vertices.}\\
    If there exist a 3-SAT solution, then there exists an assignment such that all clauses are satisfied. This assignment is analogous to a coloring scheme. Since each variable and its negation are 
    connected in the graph, they must be assigned different colors in accordance to their 3-SAT assignment. Additionally, since each variable
    is connected to either $v_{TRUE}$ or $v_{FALSE}$, the colors of the vertices must satisfy the 3-COLORING constraints.\\
    \textbf{Proof $\leftarrow$: } \textbf{If $\exists$ a 3-COLORING solution $\Rightarrow ~\exists$ a 3-SAT solution.}\\
    If there exist a valid 3-COLORING of the graph, we can use the colorings to derive a satisfying assignment to the 3-SAT problem. Since each each vertex that is connected has 
    has a different color assignment, we can assign $True$ to the variables with same color as $v_{TRUE}$, $False$ to the ones 
    with same color as $v_{FALSE}$.
\end{solution}
\end{subparts}
\end{document}